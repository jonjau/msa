\documentclass[11pt]{article}

\usepackage{geometry}
\usepackage{setspace}
\usepackage{parskip}
\usepackage{hyperref}

\geometry{a4paper, margin=20mm}

\setlength{\parindent}{0pt}

\renewcommand{\familydefault}{\sfdefault}

\begin{document}
\Large{Categorizing News Articles with Machine Learning} \\[2mm]
\large\textbf{Jonathan Jauhari --- jonathan.c.jauhari@gmail.com} \\[2mm]
\url{https://github.com/jonjau/msa2020-ml-project}

\subsection*{Aims}

\begin{itemize}
    \item To develop a news headline text classifier.
    \item To compare 3 machine learning approaches (Naive Bayes, SGD, neural 
          network) for medium-sized text classification.
    \item To roughly estimate what kind of articles the New York Times has been
          publishing in 2020 (expecting a spike in health-related articles
          due to the pandemic).
\end{itemize}

I chose the topic of classifying articles, because text identification
problems are more easily generalisable. The preprocessing steps and models
would only require few adjustments to work on other medium sized documents.
News classification in particular is intuitive in that it is easy to type in
a random news headline to feed as input to the model. This makes it simple to
test and perform analyses with.

\subsection*{Models used}

Three models were compared before deciding on which one to use in the New
York Times analysis (the detailed results of the comparison are listed in the
README):

\begin{itemize}
    \item Multinomial Naive Bayes classifier
    \item Stochastic Gradient Descent (SGD) classifier
    \item Neural network classifier with 2 hidden layers (16 nodes each)
\end{itemize}

The (Multinomial) Naive Bayes is well-known to be a great baseline model for
text identification (its assumption of feature independence is more or less
true for tokens in a document). The SGD classifier is based on
optimising gradients, and is also a model that is recommended by the
scikit-learn library for text classification.

Ultimately, however, the neural network approach proved to be the most
accurate, at the cost of higher training and preprocessing time. The neural
network model was therefore chosen for the New York Times headlines analysis.

\subsection*{Datasets used}

\begin{itemize}
\item
\href{https://www.kaggle.com/uciml/news-aggregator-dataset}
{News Aggregator Dataset}:
Headlines and categories of 400k news stories from 2014,
derived from the UCI Machine Learning Repository
\href{http://archive.ics.uci.edu/ml/datasets/News+Aggregator}{dataset}.
\item
The New York Times 2020 headlines of monthly free to read articles that I
gathered from
\href{https://spiderbites.nytimes.com/2020/}
{their site map}.
\end{itemize}

The News Aggregator Dataset was chosen, among the many \textbf{labeled} news
article datasets that are available, because it had by far the most entries
when compared to other datasets. This is likely why all the models trained on
it had high accuracies. Though it only had a few categories, each category
was broad enough to be sufficient for the analysis (a rough estimate). It
also did not require as much preprocessing as some other datasets, such as
the Reuters news dataset, as it was well documented, and already in a CSV
format, with few missing values.

The New York Times was chosen as the news website to analyse because their
site map was one of the most readily accessible. They can be taken
to represent the state of news outlets in general since they report on a
large variety of topics and cater to a global audience.

\end{document}